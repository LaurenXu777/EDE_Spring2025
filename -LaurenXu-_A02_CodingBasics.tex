% Options for packages loaded elsewhere
\PassOptionsToPackage{unicode}{hyperref}
\PassOptionsToPackage{hyphens}{url}
%
\documentclass[
]{article}
\usepackage{amsmath,amssymb}
\usepackage{iftex}
\ifPDFTeX
  \usepackage[T1]{fontenc}
  \usepackage[utf8]{inputenc}
  \usepackage{textcomp} % provide euro and other symbols
\else % if luatex or xetex
  \usepackage{unicode-math} % this also loads fontspec
  \defaultfontfeatures{Scale=MatchLowercase}
  \defaultfontfeatures[\rmfamily]{Ligatures=TeX,Scale=1}
\fi
\usepackage{lmodern}
\ifPDFTeX\else
  % xetex/luatex font selection
\fi
% Use upquote if available, for straight quotes in verbatim environments
\IfFileExists{upquote.sty}{\usepackage{upquote}}{}
\IfFileExists{microtype.sty}{% use microtype if available
  \usepackage[]{microtype}
  \UseMicrotypeSet[protrusion]{basicmath} % disable protrusion for tt fonts
}{}
\makeatletter
\@ifundefined{KOMAClassName}{% if non-KOMA class
  \IfFileExists{parskip.sty}{%
    \usepackage{parskip}
  }{% else
    \setlength{\parindent}{0pt}
    \setlength{\parskip}{6pt plus 2pt minus 1pt}}
}{% if KOMA class
  \KOMAoptions{parskip=half}}
\makeatother
\usepackage{xcolor}
\usepackage[margin=2.54cm]{geometry}
\usepackage{color}
\usepackage{fancyvrb}
\newcommand{\VerbBar}{|}
\newcommand{\VERB}{\Verb[commandchars=\\\{\}]}
\DefineVerbatimEnvironment{Highlighting}{Verbatim}{commandchars=\\\{\}}
% Add ',fontsize=\small' for more characters per line
\usepackage{framed}
\definecolor{shadecolor}{RGB}{248,248,248}
\newenvironment{Shaded}{\begin{snugshade}}{\end{snugshade}}
\newcommand{\AlertTok}[1]{\textcolor[rgb]{0.94,0.16,0.16}{#1}}
\newcommand{\AnnotationTok}[1]{\textcolor[rgb]{0.56,0.35,0.01}{\textbf{\textit{#1}}}}
\newcommand{\AttributeTok}[1]{\textcolor[rgb]{0.13,0.29,0.53}{#1}}
\newcommand{\BaseNTok}[1]{\textcolor[rgb]{0.00,0.00,0.81}{#1}}
\newcommand{\BuiltInTok}[1]{#1}
\newcommand{\CharTok}[1]{\textcolor[rgb]{0.31,0.60,0.02}{#1}}
\newcommand{\CommentTok}[1]{\textcolor[rgb]{0.56,0.35,0.01}{\textit{#1}}}
\newcommand{\CommentVarTok}[1]{\textcolor[rgb]{0.56,0.35,0.01}{\textbf{\textit{#1}}}}
\newcommand{\ConstantTok}[1]{\textcolor[rgb]{0.56,0.35,0.01}{#1}}
\newcommand{\ControlFlowTok}[1]{\textcolor[rgb]{0.13,0.29,0.53}{\textbf{#1}}}
\newcommand{\DataTypeTok}[1]{\textcolor[rgb]{0.13,0.29,0.53}{#1}}
\newcommand{\DecValTok}[1]{\textcolor[rgb]{0.00,0.00,0.81}{#1}}
\newcommand{\DocumentationTok}[1]{\textcolor[rgb]{0.56,0.35,0.01}{\textbf{\textit{#1}}}}
\newcommand{\ErrorTok}[1]{\textcolor[rgb]{0.64,0.00,0.00}{\textbf{#1}}}
\newcommand{\ExtensionTok}[1]{#1}
\newcommand{\FloatTok}[1]{\textcolor[rgb]{0.00,0.00,0.81}{#1}}
\newcommand{\FunctionTok}[1]{\textcolor[rgb]{0.13,0.29,0.53}{\textbf{#1}}}
\newcommand{\ImportTok}[1]{#1}
\newcommand{\InformationTok}[1]{\textcolor[rgb]{0.56,0.35,0.01}{\textbf{\textit{#1}}}}
\newcommand{\KeywordTok}[1]{\textcolor[rgb]{0.13,0.29,0.53}{\textbf{#1}}}
\newcommand{\NormalTok}[1]{#1}
\newcommand{\OperatorTok}[1]{\textcolor[rgb]{0.81,0.36,0.00}{\textbf{#1}}}
\newcommand{\OtherTok}[1]{\textcolor[rgb]{0.56,0.35,0.01}{#1}}
\newcommand{\PreprocessorTok}[1]{\textcolor[rgb]{0.56,0.35,0.01}{\textit{#1}}}
\newcommand{\RegionMarkerTok}[1]{#1}
\newcommand{\SpecialCharTok}[1]{\textcolor[rgb]{0.81,0.36,0.00}{\textbf{#1}}}
\newcommand{\SpecialStringTok}[1]{\textcolor[rgb]{0.31,0.60,0.02}{#1}}
\newcommand{\StringTok}[1]{\textcolor[rgb]{0.31,0.60,0.02}{#1}}
\newcommand{\VariableTok}[1]{\textcolor[rgb]{0.00,0.00,0.00}{#1}}
\newcommand{\VerbatimStringTok}[1]{\textcolor[rgb]{0.31,0.60,0.02}{#1}}
\newcommand{\WarningTok}[1]{\textcolor[rgb]{0.56,0.35,0.01}{\textbf{\textit{#1}}}}
\usepackage{graphicx}
\makeatletter
\def\maxwidth{\ifdim\Gin@nat@width>\linewidth\linewidth\else\Gin@nat@width\fi}
\def\maxheight{\ifdim\Gin@nat@height>\textheight\textheight\else\Gin@nat@height\fi}
\makeatother
% Scale images if necessary, so that they will not overflow the page
% margins by default, and it is still possible to overwrite the defaults
% using explicit options in \includegraphics[width, height, ...]{}
\setkeys{Gin}{width=\maxwidth,height=\maxheight,keepaspectratio}
% Set default figure placement to htbp
\makeatletter
\def\fps@figure{htbp}
\makeatother
\setlength{\emergencystretch}{3em} % prevent overfull lines
\providecommand{\tightlist}{%
  \setlength{\itemsep}{0pt}\setlength{\parskip}{0pt}}
\setcounter{secnumdepth}{-\maxdimen} % remove section numbering
\ifLuaTeX
  \usepackage{selnolig}  % disable illegal ligatures
\fi
\usepackage{bookmark}
\IfFileExists{xurl.sty}{\usepackage{xurl}}{} % add URL line breaks if available
\urlstyle{same}
\hypersetup{
  pdftitle={Assignment 2: Coding Basics},
  pdfauthor={Lauren Xu},
  hidelinks,
  pdfcreator={LaTeX via pandoc}}

\title{Assignment 2: Coding Basics}
\author{Lauren Xu}
\date{}

\begin{document}
\maketitle

\subsection{OVERVIEW}\label{overview}

This exercise accompanies the lessons/labs in Environmental Data
Analytics on coding basics.

\subsection{Directions}\label{directions}

\begin{enumerate}
\def\labelenumi{\arabic{enumi}.}
\tightlist
\item
  Rename this file
  \texttt{\textless{}FirstLast\textgreater{}\_A02\_CodingBasics.Rmd}
  (replacing \texttt{\textless{}FirstLast\textgreater{}} with your first
  and last name).
\item
  Change ``Student Name'' on line 3 (above) with your name.
\item
  Work through the steps, \textbf{creating code and output} that fulfill
  each instruction.
\item
  Be sure to \textbf{answer the questions} in this assignment document.
\item
  When you have completed the assignment, \textbf{Knit} the text and
  code into a single PDF file.
\item
  After Knitting, submit the completed exercise (PDF file) to Canvas.
\end{enumerate}

\subsection{Basics, Part 1}\label{basics-part-1}

\begin{enumerate}
\def\labelenumi{\arabic{enumi}.}
\item
  Generate a sequence of numbers from one to 55, increasing by fives.
  Assign this sequence a name.
\item
  Compute the mean and median of this sequence.
\item
  Ask R to determine whether the mean is greater than the median.
\item
  Insert comments in your code to describe what you are doing.
\end{enumerate}

\begin{Shaded}
\begin{Highlighting}[]
\CommentTok{\#1.Creating the data vector}
\NormalTok{vector1 }\OtherTok{\textless{}{-}} \FunctionTok{seq}\NormalTok{(}\AttributeTok{from =} \DecValTok{1}\NormalTok{, }\AttributeTok{to =} \DecValTok{55}\NormalTok{, }\AttributeTok{by =} \DecValTok{5}\NormalTok{)}
\NormalTok{vector1}
\end{Highlighting}
\end{Shaded}

\begin{verbatim}
##  [1]  1  6 11 16 21 26 31 36 41 46 51
\end{verbatim}

\begin{Shaded}
\begin{Highlighting}[]
\CommentTok{\#2.Calculating the mean and median}
\FunctionTok{mean}\NormalTok{(vector1)}
\end{Highlighting}
\end{Shaded}

\begin{verbatim}
## [1] 26
\end{verbatim}

\begin{Shaded}
\begin{Highlighting}[]
\FunctionTok{median}\NormalTok{(vector1)}
\end{Highlighting}
\end{Shaded}

\begin{verbatim}
## [1] 26
\end{verbatim}

\begin{Shaded}
\begin{Highlighting}[]
\CommentTok{\#3.Comparing the mean and median}
\FunctionTok{mean}\NormalTok{(vector1)}\SpecialCharTok{{-}}\FunctionTok{median}\NormalTok{(vector1)}
\end{Highlighting}
\end{Shaded}

\begin{verbatim}
## [1] 0
\end{verbatim}

\subsection{Basics, Part 2}\label{basics-part-2}

\begin{enumerate}
\def\labelenumi{\arabic{enumi}.}
\setcounter{enumi}{4}
\item
  Create three vectors, each with four components, consisting of (a)
  student names, (b) test scores, and (c) whether they are on
  scholarship or not (TRUE or FALSE).
\item
  Label each vector with a comment on what type of vector it is.
\item
  Combine each of the vectors into a data frame. Assign the data frame
  an informative name.
\item
  Label the columns of your data frame with informative titles.
\end{enumerate}

\begin{Shaded}
\begin{Highlighting}[]
\CommentTok{\#5.}
\CommentTok{\# Vector\_a: Student names}
\NormalTok{vector\_a }\OtherTok{\textless{}{-}} \FunctionTok{c}\NormalTok{(}\StringTok{"Ann"}\NormalTok{, }\StringTok{"Bob"}\NormalTok{, }\StringTok{"Eric"}\NormalTok{, }\StringTok{"David"}\NormalTok{) }
\CommentTok{\# Vector\_b: Test scores}
\NormalTok{vector\_b }\OtherTok{\textless{}{-}} \FunctionTok{c}\NormalTok{(}\DecValTok{77}\NormalTok{, }\DecValTok{88}\NormalTok{, }\DecValTok{78}\NormalTok{, }\DecValTok{66}\NormalTok{)}
\CommentTok{\# Vector\_c: whether they are on scholarship or not (TRUE or FALSE)}
\NormalTok{vector\_c }\OtherTok{\textless{}{-}} \FunctionTok{c}\NormalTok{(}\ConstantTok{TRUE}\NormalTok{, }\ConstantTok{TRUE}\NormalTok{, }\ConstantTok{TRUE}\NormalTok{, }\ConstantTok{FALSE}\NormalTok{)}

\CommentTok{\#7.}
\NormalTok{df\_vector\_a }\OtherTok{\textless{}{-}} \FunctionTok{as.data.frame}\NormalTok{(vector\_a)}
\NormalTok{df\_vector\_b }\OtherTok{\textless{}{-}} \FunctionTok{as.data.frame}\NormalTok{(vector\_b)}
\NormalTok{df\_vector\_c }\OtherTok{\textless{}{-}} \FunctionTok{as.data.frame}\NormalTok{(vector\_c)}
\NormalTok{df\_ScholarshipList }\OtherTok{\textless{}{-}} \FunctionTok{cbind}\NormalTok{(df\_vector\_a,df\_vector\_b,df\_vector\_c)}
\NormalTok{df\_ScholarshipList}
\end{Highlighting}
\end{Shaded}

\begin{verbatim}
##   vector_a vector_b vector_c
## 1      Ann       77     TRUE
## 2      Bob       88     TRUE
## 3     Eric       78     TRUE
## 4    David       66    FALSE
\end{verbatim}

\begin{Shaded}
\begin{Highlighting}[]
\CommentTok{\#8.}
\FunctionTok{names}\NormalTok{(df\_ScholarshipList) }\OtherTok{\textless{}{-}} \FunctionTok{c}\NormalTok{(}\StringTok{\textquotesingle{}StudentName\textquotesingle{}}\NormalTok{, }\StringTok{\textquotesingle{}TestScore\textquotesingle{}}\NormalTok{, }\StringTok{\textquotesingle{}IfScholarship\textquotesingle{}}\NormalTok{)}
\NormalTok{df\_ScholarshipList}
\end{Highlighting}
\end{Shaded}

\begin{verbatim}
##   StudentName TestScore IfScholarship
## 1         Ann        77          TRUE
## 2         Bob        88          TRUE
## 3        Eric        78          TRUE
## 4       David        66         FALSE
\end{verbatim}

\begin{enumerate}
\def\labelenumi{\arabic{enumi}.}
\setcounter{enumi}{8}
\tightlist
\item
  QUESTION: How is this data frame different from a matrix?
\end{enumerate}

\begin{quote}
Answer: A data frame can have different data types in columns. But
matrix can only contain a single data type.
\end{quote}

\begin{enumerate}
\def\labelenumi{\arabic{enumi}.}
\setcounter{enumi}{9}
\item
  Create a function with one input. In this function, use
  \texttt{if}\ldots{}\texttt{else} to evaluate the value of the input:
  if it is greater than 50, print the word ``Pass''; otherwise print the
  word ``Fail''.
\item
  Create a second function that does the exact same thing as the
  previous one but uses \texttt{ifelse()} instead if
  \texttt{if}\ldots{}\texttt{else}.
\item
  Run both functions using the value 52.5 as the input
\item
  Run both functions using the \textbf{vector} of student test scores
  you created as the input. (Only one will work properly\ldots)
\end{enumerate}

\begin{Shaded}
\begin{Highlighting}[]
\CommentTok{\#10. Create a function using if...else}
\NormalTok{check\_score1 }\OtherTok{\textless{}{-}} \ControlFlowTok{function}\NormalTok{(score) \{}
  \ControlFlowTok{if}\NormalTok{ (score }\SpecialCharTok{\textgreater{}} \DecValTok{70}\NormalTok{) \{}
    \StringTok{"Pass"}
\NormalTok{  \} }\ControlFlowTok{else}\NormalTok{ \{}
    \StringTok{"Fail"}
\NormalTok{  \}}
\NormalTok{\}}

\CommentTok{\#11. Create a function using ifelse()}
\NormalTok{check\_score2 }\OtherTok{\textless{}{-}} \ControlFlowTok{function}\NormalTok{(score) \{}
  \FunctionTok{ifelse}\NormalTok{(score }\SpecialCharTok{\textgreater{}} \DecValTok{70}\NormalTok{, }\StringTok{"Pass"}\NormalTok{, }\StringTok{"Fail"}\NormalTok{)}
\NormalTok{\}}

\CommentTok{\#12a. Run the first function with the value 52.5}
\FunctionTok{check\_score1}\NormalTok{(}\FloatTok{52.5}\NormalTok{)}
\end{Highlighting}
\end{Shaded}

\begin{verbatim}
## [1] "Fail"
\end{verbatim}

\begin{Shaded}
\begin{Highlighting}[]
\CommentTok{\#12b. Run the second function with the value 52.5}
\FunctionTok{check\_score2}\NormalTok{(}\FloatTok{52.5}\NormalTok{)}
\end{Highlighting}
\end{Shaded}

\begin{verbatim}
## [1] "Fail"
\end{verbatim}

\begin{Shaded}
\begin{Highlighting}[]
\CommentTok{\#13a. Run the first function with the vector of test scores}
\CommentTok{\#vector\_b \textless{}{-} c(77, 88, 78, 66)}
\CommentTok{\#check\_score1(vector\_b)}

\CommentTok{\#13b. Run the second function with the vector of test scores}
\FunctionTok{check\_score2}\NormalTok{(vector\_b)}
\end{Highlighting}
\end{Shaded}

\begin{verbatim}
## [1] "Pass" "Pass" "Pass" "Fail"
\end{verbatim}

\begin{enumerate}
\def\labelenumi{\arabic{enumi}.}
\setcounter{enumi}{13}
\tightlist
\item
  QUESTION: Which option of \texttt{if}\ldots{}\texttt{else}
  vs.~\texttt{ifelse} worked? Why? (Hint: search the web for ``R
  vectorization'')
\end{enumerate}

\begin{quote}
Answer: Only `ifelse' worked, whereas `if'\ldots{}`else' dose not
worked. Because `if'\ldots{}`else' only apply to a single value rather
than a vector.
\end{quote}

\textbf{NOTE} Before knitting, you'll need to comment out the call to
the function in Q13 that does not work. (A document can't knit if the
code it contains causes an error!)

\end{document}
